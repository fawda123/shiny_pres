\documentclass[serif]{beamer}\usepackage[]{graphicx}\usepackage[]{color}
%% maxwidth is the original width if it is less than linewidth
%% otherwise use linewidth (to make sure the graphics do not exceed the margin)
\makeatletter
\def\maxwidth{ %
  \ifdim\Gin@nat@width>\linewidth
    \linewidth
  \else
    \Gin@nat@width
  \fi
}
\makeatother

\definecolor{fgcolor}{rgb}{0.345, 0.345, 0.345}
\newcommand{\hlnum}[1]{\textcolor[rgb]{0.686,0.059,0.569}{#1}}%
\newcommand{\hlstr}[1]{\textcolor[rgb]{0.192,0.494,0.8}{#1}}%
\newcommand{\hlcom}[1]{\textcolor[rgb]{0.678,0.584,0.686}{\textit{#1}}}%
\newcommand{\hlopt}[1]{\textcolor[rgb]{0,0,0}{#1}}%
\newcommand{\hlstd}[1]{\textcolor[rgb]{0.345,0.345,0.345}{#1}}%
\newcommand{\hlkwa}[1]{\textcolor[rgb]{0.161,0.373,0.58}{\textbf{#1}}}%
\newcommand{\hlkwb}[1]{\textcolor[rgb]{0.69,0.353,0.396}{#1}}%
\newcommand{\hlkwc}[1]{\textcolor[rgb]{0.333,0.667,0.333}{#1}}%
\newcommand{\hlkwd}[1]{\textcolor[rgb]{0.737,0.353,0.396}{\textbf{#1}}}%

\usepackage{framed}
\makeatletter
\newenvironment{kframe}{%
 \def\at@end@of@kframe{}%
 \ifinner\ifhmode%
  \def\at@end@of@kframe{\end{minipage}}%
  \begin{minipage}{\columnwidth}%
 \fi\fi%
 \def\FrameCommand##1{\hskip\@totalleftmargin \hskip-\fboxsep
 \colorbox{shadecolor}{##1}\hskip-\fboxsep
     % There is no \\@totalrightmargin, so:
     \hskip-\linewidth \hskip-\@totalleftmargin \hskip\columnwidth}%
 \MakeFramed {\advance\hsize-\width
   \@totalleftmargin\z@ \linewidth\hsize
   \@setminipage}}%
 {\par\unskip\endMakeFramed%
 \at@end@of@kframe}
\makeatother

\definecolor{shadecolor}{rgb}{.97, .97, .97}
\definecolor{messagecolor}{rgb}{0, 0, 0}
\definecolor{warningcolor}{rgb}{1, 0, 1}
\definecolor{errorcolor}{rgb}{1, 0, 0}
\newenvironment{knitrout}{}{} % an empty environment to be redefined in TeX

\usepackage{alltt}
\usetheme{EPA}
\usepackage{graphicx}
\usepackage{xcolor}
\usepackage{tikz}
\usetikzlibrary{shadows,arrows,positioning}

\newcommand{\emtxt}[1]{\textbf{\textit{#1}}}

\tikzstyle{block} = [rectangle, draw, text width=7em, text centered, rounded corners, minimum height=3em, minimum width=7em, top color = white, bottom color=brown!30,  drop shadow]

% knitr setup


\IfFileExists{upquote.sty}{\usepackage{upquote}}{}
\begin{document}

\title[Shiny Overview]{\textbf{An overview of Shiny applications using R and RStudio}\vspace{-0.15in}}
\author[M. Beck]{Marcus W. Beck\inst{1}}

\institute[USEPA]{\inst{1} USEPA National Health and Environmental Effects Research Laboratory, Gulf Ecology Division, \href{mailto:beck.marcus@epa.gov}{beck.marcus@epa.gov}}

\date{Dec. 10, 2015}

%%%%%%
\begin{frame}
\titlepage
\end{frame}

%%%%%%
\begin{frame}{Who am I?}
\begin{itemize}
\item ORISE post-doc for 2.5 years, fed postdoc since last week \\~\\
\item NHEERL Gulf Ecology Division \\~\\
\item Research focus on water quality assessment and indicator development \\~\\
\item Specific interests in statistical modelling, data assimilation, graphics  \\~\\
\end{itemize}
\end{frame}

%%%%%%
\begin{frame}{Who am I?}
R user since 2007 \\~\\
Maintainer of two packages on CRAN: \\~\\
\begin{columns}[T]
\begin{column}{0.45\textwidth}
\emtxt{SWMPr}\\~\\
Tools for retrieving, organizing, and analyzing environmental data from the System Wide Monitoring Program of the National Estuarine Research Reserve System. 
\end{column}
\begin{column}{0.45\textwidth}
\emtxt{NeuralNetTools} \\~\\
Visualization and analysis tools to aid in the interpretation of neural network models
\end{column}
\end{columns}
\end{frame}

%%%%%%
\begin{frame}{Reproducible research workflow}
General workflow for \emtxt{reproducible research} - reproduce results from an experiment or analysis conducted by another.\\~\\
From Wikipedia... `The ultimate product is the \emtxt{paper along with the full computational environment} used to produce the results in the paper such as the code, data, etc. that can be \emtxt{used to reproduce the results and create new work} based on the research.'\\~\\
\begin{columns}
\begin{column}{0.25\textwidth}
\centerline{\includegraphics[width = \textwidth]{fig/Rlogo.png}}
\end{column}
\begin{column}{0.25\textwidth}
\centerline{\includegraphics[width = \textwidth]{fig/RStudio.png}}
\end{column}
\begin{column}{0.25\textwidth}
\centerline{\includegraphics[width = \textwidth]{fig/knit-logo.png}}
\end{column}
\begin{column}{0.25\textwidth}
\centerline{\includegraphics[width = \textwidth]{fig/octocat.png}}
\end{column}
\end{columns}
\end{frame}

%%%%%%
\begin{frame}{Reproducible research workflow}
\begin{columns}
\begin{column}{0.23\textwidth}
\centerline{\includegraphics[width = \textwidth]{fig/Rlogo.png}}
\end{column}
\begin{column}{0.23\textwidth}
\centerline{\includegraphics[width = \textwidth]{fig/RStudio.png}}
\end{column}
\begin{column}{0.23\textwidth}
\centerline{\includegraphics[width = \textwidth]{fig/knit-logo.png}}
\end{column}
\begin{column}{0.23\textwidth}
\centerline{\includegraphics[width = \textwidth]{fig/octocat.png}}
\end{column}
\end{columns}
\vspace{0.2in}
The use of these tools increases transparency and transfer of information \emtxt{= better science}\\~\\
Data prep, analysis, report, and sharing can all be done in RStudio IDE
\end{frame}

%%%%%%
\begin{frame}{Introduction to Shiny}
Where does Shiny fit with reproducible research? \\~\\
Shiny is a web application framework for R \\~\\
\begin{columns}
\begin{column}{0.7\textwidth}
\begin{itemize}
\item From the command line to a user interface 
\item Make your code interactive 
\item Do not need to know anything about web programming
\item Integrated very well with R studio \\~\\
\end{itemize}
\end{column}
\begin{column}{0.2\textwidth}
\centerline{\includegraphics[width = \textwidth]{fig/shiny_logo.png}}
\end{column}
\end{columns}
\vspace{0.16in}
Tools like Shiny improve \emtxt{accessibility} and \emtxt{communication} 
\end{frame}

\end{document}
